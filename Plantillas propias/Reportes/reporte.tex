\documentclass[
    11pt, 
    leqno, 
    twoside, 
    b5paper,
    mpinclude=true,
    abstract=true
    ]{scrartcl}

%%%%%%%%%%%%%%%%%%%%%%%%%%%%%%%%%%%%%%%%%%%%%%%%%%%%%%%%%%%%%%%%%%%%%%%%%%%%%%%%%
% Opciones de inputs (útil si no se compila con LuaLaTeX o XeLaTeX)
%\usepackage[utf8]{inputenc}

%%%%%%%%%%%%%%%%%%%%%%%%%%%%%%%%%%%%%%%%%%%%%%%%%%%%%%%%%%%%%%%%%%%%%%%%%%%%%%%%%
% Idioma del documento
\usepackage[spanish,es-nodecimaldot,es-tabla,es-lcroman]{babel}

%%%%%%%%%%%%%%%%%%%%%%%%%%%%%%%%%%%%%%%%%%%%%%%%%%%%%%%%%%%%%%%%%%%%%%%%%%%%%%%%%
% Permitir más alfabetos matemáticos
\newcommand\hmmax{0}
\newcommand\bmmax{0}

%%%%%%%%%%%%%%%%%%%%%%%%%%%%%%%%%%%%%%%%%%%%%%%%%%%%%%%%%%%%%%%%%%%%%%%%%%%%%%%%%
% Carga el archivo macros.sty que tiene definiciones que se usan más adelante
\usepackage{macros}

%%%%%%%%%%%%%%%%%%%%%%%%%%%%%%%%%%%%%%%%%%%%%%%%%%%%%%%%%%%%%%%%%%%%%%%%%%%%%%%%%
% Opciones de la fuente pdfLaTeX
%\usepackage{fouriernc}
%\usepackage[scale=0.9]{plex-serif}
%\usepackage[scaled=0.85]{FiraSans}

%%%%%%%%%%%%%%%%%%%%%%%%%%%%%%%%%%%%%%%%%%%%%%%%%%%%%%%%%%%%%%%%%%%%%%%%%%%%%%%%%
% Fuente con LuaLaTex o XeLaTeX, preferentemente LuaLaTeX por microtype
\usepackage{fontspec}
\usepackage{fourier-otf}
\usepackage[
    DefaultFeatures={Scale=MatchLowercase},
    RMstyle={Text,Semibold},
    SSstyle={Text,Semibold},
    SSconstyle={Text,Semibold},
    TTstyle={Text,Semibold}]{plex-otf}
\setsansfont{FiraSans}[Scale=MatchLowercase]
\setmathfont{Erewhon Math}[Style=leqslant,Scale=MatchUppercase]

%%%%%%%%%%%%%%%%%%%%%%%%%%%%%%%%%%%%%%%%%%%%%%%%%%%%%%%%%%%%%%%%%%%%%%%%%%%%%%%%%
% Datos del trabajo
\newcommand{\trabajo}{Título del reporte}
\newcommand{\materia}{Mi materia favorita}
\newcommand{\fechaentrega}{31 de febrero de 2371}
\newcommand{\minombre}{Mi nombre, supongo}
\author{\minombre}
\title{\trabajo}
\date{\fechaentrega}

%%%%%%%%%%%%%%%%%%%%%%%%%%%%%%%%%%%%%%%%%%%%%%%%%%%%%%%%%%%%%%%%%%%%%%%%%%%%%%%%%
% Paquetería para floats: 
\usepackage{graphicx}
\usepackage{booktabs}
% Quitar el número de la sección de los contadores de figuras y tablas
\counterwithout{figure}{section}
\counterwithout{table}{section}
% Fuente de las palabras Figura y Tabla
\setkomafont{captionlabel}{\sffamily}

%%%%%%%%%%%%%%%%%%%%%%%%%%%%%%%%%%%%%%%%%%%%%%%%%%%%%%%%%%%%%%%%%%%%%%%%%%%%%%%%%
% Márgenes
\usepackage[margin=2cm]{geometry}

%%%%%%%%%%%%%%%%%%%%%%%%%%%%%%%%%%%%%%%%%%%%%%%%%%%%%%%%%%%%%%%%%%%%%%%%%%%%%%%%%
% Formato de la encabezados y pies de página
\usepackage[automark]{scrlayer-scrpage}% sets pagestyle scrheadings automatically
\clearpairofpagestyles
\ofoot*{\pagemark}
\rehead*[]{\trabajo}
\lehead*[]{\headmark}
\lohead*[]{\materia}
\rohead*[]{\headmark}

% Pie de página con líneas para separar el encabezado y el pie de página
% del resto
\KOMAoptions{
  footsepline=1pt:.25\paperwidth,% syntax: footsepline=<thickness>:<length>
  plainfootsepline,
  olines,
  headsepline=0.5pt:\textwidth
}
% Recorre la línea a la parte exterior de la hoja
\ModifyLayer[
  hoffset=0pt,
  width=\paperwidth,
  addvoffset=-5pt% move the line up
]{scrheadings.foot.above.line}
\RedeclareLayer[
  clone=scrheadings.foot.above.line
]{plain.scrheadings.foot.above.line}

% Fuente de pies de página y encabezados
\setkomafont{pagenumber}{\normalfont\sffamily\bfseries}
\setkomafont{pagehead}{\sffamily}
\renewcommand*{\sectionmarkformat}{}
\automark*[section]{section}

%%%%%%%%%%%%%%%%%%%%%%%%%%%%%%%%%%%%%%%%%%%%%%%%%%%%%%%%%%%%%%%%%%%%%%%%%%%%%%%%%
% Paquetería adicional
\usepackage{microtype}
\usepackage{multicol}
\usepackage{anyfontsize}
\usepackage{ragged2e}
\usepackage{comment}
\usepackage{pdfpages}
\usepackage{lipsum}

%%%%%%%%%%%%%%%%%%%%%%%%%%%%%%%%%%%%%%%%%%%%%%%%%%%%%%%%%%%%%%%%%%%%%%%%%%%%%%%%%
% Opciones de enlistados
\usepackage[inline]{enumitem}
% Los enlistados ahora tienen (i), (ii), ... en el primer nivel y (a), (b), ...
% en el segundo nivel
\setlist[1]{parsep=0pt,
topsep=3pt,
itemsep=0pt,
label=\textup{(\roman*)},
leftmargin=*}
\setlist[2]{parsep=0pt,
topsep=3pt,
itemsep=0pt,
label=\textup{(\alph*)}, 
leftmargin=*}

%%%%%%%%%%%%%%%%%%%%%%%%%%%%%%%%%%%%%%%%%%%%%%%%%%%%%%%%%%%%%%%%%%%%%%%%%%%%%%%%%
% Redefinir el ambiente de demostración para que la fuente de la palabra 
% Demostración sea sans

\usepackage{etoolbox}
\makeatletter

\renewenvironment{proof}[1][\proofname]{\par
  \pushQED{\qed}%
  \normalfont \topsep6\p@\@plus6\p@\relax
  \trivlist\item\relax{
    \sffamily#1\@addpunct{.}}\hspace\labelsep\ignorespaces
  }{%
  \popQED\endtrivlist\@endpefalse}
\makeatother
% Redefinir el símbolo al final de una demostración
\renewcommand{\qedsymbol}{$\blacksquare$}

%%%%%%%%%%%%%%%%%%%%%%%%%%%%%%%%%%%%%%%%%%%%%%%%%%%%%%%%%%%%%%%%%%%%%%%%%%%%%%%%%
% TikZ y PGFPLots
\usepackage{tikz}
\usetikzlibrary{
    knots,
    hobby,
    decorations.pathreplacing,
    shapes.geometric,
    calc,
    shapes
    }
\usepackage{pgfplots}
\pgfplotsset{compat=1.7}
\usepackage{pgfornament}

%%%%%%%%%%%%%%%%%%%%%%%%%%%%%%%%%%%%%%%%%%%%%%%%%%%%%%%%%%%%%%%%%%%%%%%%%%%%%%%%%
% Bibliografía con BibLaTeX
\usepackage[backend=biber,style=apa]{biblatex}
\usepackage{csquotes}
% Usamos el formato APA
\DeclareLanguageMapping{spanish}{spanish-apa}
\urlstyle{same}
% Cargamos la fuente con los datos bibliográficos
\addbibresource{demo.bib}

%%%%%%%%%%%%%%%%%%%%%%%%%%%%%%%%%%%%%%%%%%%%%%%%%%%%%%%%%%%%%%%%%%%%%%%%%%%%%%%%%
% Hipervínculos y enlaces dentro del documento
% Definición de colores
\definecolor{mytitlecolor}{RGB}{115,35,60}
\definecolor{mysectioncolor}{RGB}{104,34,139}
\usepackage{hyperref}
\hypersetup{
        colorlinks=true,
        linkcolor=mysectioncolor,
        filecolor=magenta,      
        urlcolor=cyan,
        pdfauthor={el autor},
    citecolor=mysectioncolor
}
\usepackage[spanish]{cleveref}

\begin{document}

\begin{titlepage}
    \begin{addmargin}[-1.5cm]{-1.5cm}
        \centering % centrada
        %\raggedleft % Alineada a la derecha
        %\raggedright % Alineada a la izquierda
        
        %\vspace*{\baselineskip} % Línea vertical 
        %\begin{picture}(0,0)
        %  \put(-7cm,-1.3\textheight){\color{coolred}\rule{0.5cm}{1.2\paperheight}}
        %\end{picture}

        \hfill\includegraphics[width=4cm]{example-image-a}
        \vspace*{0.05\textheight} % Whitespace before the title
        
        %------------------------------------------------
        %	Cosas del título
        %------------------------------------------------
            
        \color{coolred}
        \rule{0.8\textwidth}{2pt}
        \vspace*{\baselineskip}

            % Nombre de la materia  
        \textsf{\textbf{\huge \color{mytitlecolor}{\materia}}}  
        \vspace*{0.05\textheight}

        % Aquí va el título
        {\Huge{\textsf{\textbf{\color{mytitlecolor}{\trabajo}}}}}
        \vspace*{\baselineskip}

        \rule{0.8\textwidth}{2pt}

        \vspace*{0.03\textheight}

        %------------------------------------------------
        %	Aquí van los nombres
        %------------------------------------------------
        
        \color{black}
        {\large \textsf{Alumnos:}}  \vspace*{1.5\baselineskip}

        {\Large Alumno \# 1} \vspace*{\baselineskip}

        {\Large Alumno \# 2} \vspace*{\baselineskip}

        {\Large Alumno \# 3} \vspace*{2\baselineskip}

        {\large \textsf{Profesor:}} \vspace*{1.5\baselineskip}

        {\Large Nombre del profesor} \vspace*{2\baselineskip}

        {\large \textsf{Escuela, Ciudad}} \vspace*{\baselineskip}

        {\large \textsf{\fechaentrega}}
        
        \vspace*{\fill}
    \end{addmargin}
\end{titlepage}

\maketitle

\begin{abstract}
    \noindent Aquí va un pequeño resumen del trabajo si es que es necesario. De no serlo, quitar la opción \texttt{abstract=true}. Ahora solamente voy a escribir cosas para llenar espacio y que esto no se vea tan vacío.
\end{abstract}

\section{Estructura del trabajo}

\noindent Usualmente los reportes escritos tienden a ser un poco largos, por lo que no es ideal tener todo el código en un mismo archivo \texttt{.tex}. En este sentido, una de las fortalezas de \LaTeX~es poder incluir subarchivos mediante el uso de las funciones \texttt{\textbackslash{}input} e \texttt{\textbackslash{}include}. Por ejemplo, todas las secciones en este trabajo se incluyen en el archivo \texttt{reporte.tex} mediante el uso de \texttt{\textbackslash{}include\{secciones/seccion1\}}. De manera un poco más estructurada:
\begin{enumerate}
    \item Se crea el archivo principal, en este caso \texttt{reportes.tex}
    \item En el mismo directorio del archivo principal se crea un subdirectorio, en este caso llamado \texttt{secciones}, en el que se crean los archivos \texttt{.tex} correspondientes a las secciones.
    \item En los archivos correspondientes a las secciones se escribe lo que deba ser escrito.
    \item Se incluyen los archivos de las secciones en el archivo principal.
\end{enumerate}
El hacer esto hace más fácil el poder encontrar errores y modificar, pues limita de algún modo el tamaño de los \texttt{.tex}.
\section{La portada}

\noindent Como podrá verse, en este \texttt{pdf} hay una portada ``elegante'', la cual puede modificarse al gusto de cada quién. Esta portada es útil cuando el trabajo se realiza entre varias personas. También se muestra lo que sería un título normal, que puede ser usado para trabajos de una persona menos extensivos. Si en lugar de una portada se pone el título, en la primera página no habrá encabezados, porque estos no son estéticos en tal caso.
\section{Matemáticas}

\noindent Algunos ambientes matemáticos se muestran a continuación.

\begin{definition}
    Soy la definición necesaria.
\end{definition}

\begin{theorem}
    Soy el teorema importante.
\end{theorem}

Como todos saben, un teorema debe ser demostrado. A continuación se da la prueba:

\begin{proof}
    Se deja como ejercicio al lector.
\end{proof}

Si la prueba termina en una ecuación hay que usar \texttt{\textbackslash{}qedhere} para que el símbolo de que terminó la prueba salga en la misma línea que la ecuación:

\begin{proof}
    Evidentemente se cumple que 
    \[
        a^3 + b^3 = c^3. \qedhere
    \]
\end{proof}
\section{Yadayadayada}

\noindent \lipsum

\nocite{*}
\printbibliography

\end{document}