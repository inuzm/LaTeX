\documentclass[
letterpaper,
11pt, % Cambiar a 10 si es que no cabe
oneside,
onecolumn, %twocolumn para dos columnas
article
]{memoir}

\usepackage[spanish,es-nodecimaldot]{babel}
\usepackage[utf8]{inputenc}
\usepackage[T1]{fontenc}
\usepackage{tgtermes} % La fuente a usar, si no compila quitar esta línea
\usepackage[svgnames]{xcolor} % Required for colour specification
\medievalpage

% Paquetes para matemáticas
\usepackage{amscd}
\usepackage{amsfonts}
\usepackage{amssymb}
\usepackage{amsmath}
\usepackage{amsthm}
\usepackage{latexsym}
\usepackage{mathrsfs}
\usepackage{bm}
\usepackage{bbm}
\usepackage{mathtools}
\usepackage{listings}
\usepackage[spanish,onelanguage,ruled,linesnumbered]{algorithm2e}

\usepackage[final]{microtype}
\usepackage{graphicx} % Para incluir figuras
\usepackage{lipsum}

\usepackage{hyperref}

\setlrmarginsandblock{0.15\paperwidth}{*}{1} % Para onecolumn
\setulmarginsandblock{0.5in}{1.5in}{1}  % Márgenes superior e inferior
\checkandfixthelayout

\addto{\captionsspanish}{%
  \renewcommand{\bibname}{\Large Referencias}
}

\counterwithout{section}{chapter}
\counterwithout{figure}{chapter}

\makepagestyle{plain}
\makeevenfoot{plain}{\thepage}{}{}
\makeoddfoot{plain}{}{}{\thepage}
\makeevenhead{plain}{}{}{}
\makeoddhead{plain}{}{}{}

\makeatletter %
\makechapterstyle{standard}{
  \setlength{\beforechapskip}{2\baselineskip}
  \setlength{\midchapskip}{0\baselineskip}
  \setlength{\afterchapskip}{2\baselineskip}
  \renewcommand{\chapterheadstart}{\vspace*{\beforechapskip}}
  \renewcommand{\chapnamefont}{\normalfont\Large}
  \renewcommand{\printchaptername}{}
  \renewcommand{\chapternamenum}{\space}
  \renewcommand{\chapnumfont}{\normalfont\Large}
  %\renewcommand{\printchapternum}{\chapnumfont \thechapter.}
  %\renewcommand{\afterchapternum}{\par\nobreak\vskip \midchapskip}
  \renewcommand{\afterchapternum}{ }
  \renewcommand{\printchapternonum}{\vspace*{\midchapskip}\vspace*{5mm}}
  \renewcommand{\chaptitlefont}{\bfseries\LARGE}
  \renewcommand{\printchaptertitle}[1]{\chaptitlefont ##1}
  \renewcommand{\afterchaptertitle}{\par\nobreak\vskip \afterchapskip}
}
\makeatother

\chapterstyle{standard}

\makeatletter %
\makechapterstyle{appendix}{
  \setlength{\beforechapskip}{2\baselineskip}
  \setlength{\midchapskip}{0\baselineskip}
  \setlength{\afterchapskip}{2\baselineskip}
  \renewcommand{\chapterheadstart}{\vspace*{\beforechapskip}}
  \renewcommand{\chapnamefont}{\normalfont\Large}
  \renewcommand{\printchaptername}{\chapnamefont \@chapapp}
  \renewcommand{\chapternamenum}{\space}
  \renewcommand{\chapnumfont}{\normalfont\Large}
  \renewcommand{\printchapternum}{\chapnumfont \thechapter.}
  %\renewcommand{\afterchapternum}{\par\nobreak\vskip \midchapskip}
  \renewcommand{\afterchapternum}{ }
  \renewcommand{\printchapternonum}{\vspace*{\midchapskip}\vspace*{5mm}}
  \renewcommand{\chaptitlefont}{\bfseries\LARGE}
  \renewcommand{\printchaptertitle}[1]{\chaptitlefont ##1}
  \renewcommand{\afterchaptertitle}{\par\nobreak\vskip \afterchapskip}
}
\makeatother

% Se definen los comandos para escribir teoremas, definiciones y demás.
\theoremstyle{plain}
\newtheorem{theorem}{Teorema}
\newtheorem{corollary}[theorem]{Corolario}
\newtheorem{lemma}[theorem]{Lema}
\newtheorem{proposition}[theorem]{Proposici\'on}
\theoremstyle{definition}
\newtheorem{definition}[theorem]{Definici\'on}
\theoremstyle{remark}
\newtheorem{remark}[theorem]{Observaci\'on}

\begin{document}

%%%%%%%%%%%%%%%%%%%%%%%%%
% Aquí va la portada
%%%%%%%%%%%%%%%%%%%%%%%%%

\begin{titlingpage} % Portada

    \raggedleft % Alineada a la derecha
    %\raggedright % Alineada a la izquierda
	
	\vspace*{\baselineskip} % Whitespace at the top of the page
	
	\vspace*{0.25\textheight} % Whitespace before the title
	
	%------------------------------------------------
	%	Cosas del título
	%------------------------------------------------
	
    \textbf{\huge Procesos Estocásticos I}\\[\baselineskip] % Nombre de la materia
    
    \vspace*{0.1\textheight}

    {\Huge{\textbf{Ponerle título}}}\\[\baselineskip] % Aquí va el título
    \vspace*{0.1\textheight}

    %------------------------------------------------
	%	Aquí van los nombres
	%------------------------------------------------
    
    {\Large Alumno 1}\\[\baselineskip]
    {\Large Alumno 2}\\[\baselineskip]
    {\Large Alumno 3}\\[\baselineskip]
    {\Large Alumno 4}\\[\baselineskip]
    {\Large Alumno 5}\\[\baselineskip]
	
	\vfill

\end{titlingpage}

\thispagestyle{empty}

\chapter*{Cadenas de Markov} % 

% Copiar y pegar lo que tengan de la cadena de Markov

\noindent \lipsum[1-9]

Para escribir un teorema:

\begin{theorem}
    Hola
\end{theorem}

Para escribir una definición:

\begin{definition}
    Lo que sea.
\end{definition}

\newpage

\chapter*{Proceso Poisson}

% Copiar y pegar lo que tengan del proceso Poisson

\newpage

\chapter*{Martingalas}

% Copiar y pegar lo que tengan de martingalas

\newpage

\chapter*{Movimiento browniano}

% Copiar y pegar lo que tengan del movimiento browniano

\newpage

\appendix
\chapterstyle{appendix}

\chapter{Códigos}

\noindent En esta parte van todos los códigos. Un ejemplo del paquete \texttt{listings}:

\begin{lstlisting}[language=R]
  poisson_tiempo <- function(Tiempo, intensidad){
      # Numero de saltos
      N_aux <- rpois(n = 1, lambda = intensidad * Tiempo)
      # Donde ocurrieron los saltos
      # Agregamos el cero y el tiempo final
      tiempos <- c(0, sort(runif(n = N_aux, max = Tiempo)), Tiempo)
      # Regresamos data frame con los tiempos y 
      # los valores para graficar
      return(
          data.frame(
              t = tiempos,
              n = c(0:N_aux, N_aux)
          )
      )
  }
\end{lstlisting}

Si lo prefieren, pueden escribir pseudoalgoritmos con el paquete \texttt{algorithm2e}. Un ejemplo:

\begin{algorithm}[H]
  \Begin{
    Simular $X \sim h$ y $U \sim \mathrm{U}(0,1)$ \;
    \lIf{$\frac{M h(X) U}{g(X)} \leq 1$}{se acepta $X$}
    \lElse{se regresa a 2}
  }
  \KwResult{Una simulación $X \sim g$.}
  \caption{Método de aceptación y rechazo.}
\end{algorithm}

\newpage

\chapter{Imágenes}

\noindent En esta parte pueden poner imágenes.

\newpage

\backmatter

\chapter*{Referencias}

\end{document}
