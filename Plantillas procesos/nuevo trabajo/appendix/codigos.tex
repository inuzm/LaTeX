\chapter{Códigos}

\noindent En esta parte van todos los códigos. Un ejemplo del paquete \texttt{listings}:

{\small
\begin{lstlisting}[language=R]
  poisson_tiempo <- function(Tiempo, intensidad){
      # Numero de saltos
      N_aux <- rpois(n = 1, lambda = intensidad * Tiempo)
      # Donde ocurrieron los saltos
      # Agregamos el cero y el tiempo final
      tiempos <- c(0, sort(runif(n = N_aux, max = Tiempo)), Tiempo)
      # Regresamos data frame con los tiempos y 
      # los valores para graficar
      return(
          data.frame(
              t = tiempos,
              n = c(0:N_aux, N_aux)
          )
      )
  }
\end{lstlisting}
}

Si lo prefieren, pueden escribir pseudoalgoritmos con el paquete \texttt{algorithm2e}. Un ejemplo:

\begin{algorithm}[htp]
  \Begin{
    Simular $X \sim h$ y $U \sim \mathrm{U}(0,1)$ \;
    \lIf{$\frac{M h(X) U}{g(X)} \leq 1$}{se acepta $X$}
    \lElse{se regresa a 2}
  }
  \KwResult{Una simulación $X \sim g$.}
  \caption{Método de aceptación y rechazo.}
\end{algorithm}